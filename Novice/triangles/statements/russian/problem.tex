На вход программы подаются длины трех отрезков, которые могут (или нет) составить треугольник.
По входным данным необходимо определить:

\begin{itemize}
	\item Составляют ли отрезки треугольники;
	\item Если составляют, то является ли треугольник обычным, равнобедренным или равносторонним.
\end{itemize}

\InputFile

$X Y Z$~---~длины отрезков предполагаемого треугольника

\OutputFile

Необходимо вывести следующие значения:

\begin{itemize}
	\item \texttt{NOT TRIANGLE}~---~отрезки не образуют треугольник;
	\item \texttt{PLAIN TRIANGLE}~---~отрезки образуют обычный треугольник;
	\item \texttt{ISOSCELES TRIANGLE}~---~отрезки образуют \emph{равнобедренный} треугольник;
	\item \texttt{EQUILATERAL TRIANGLE}~---~отрезки образуют \textbf{равносторонний} треугольник;
\end{itemize}

\SAMPLES