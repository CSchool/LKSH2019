На вход программы подается массив размером $N$. Необходимо вывести частоту появления элементов в массиве.
\InputFile

\noindent
$N$~---~размер массива, $1 \leq N \leq 100$. \\
$a_1, a_2, \ldots a_N$~---~массив с числами. $ a_i \in [0, 100]$. 

\OutputFile

\begin{align*}
	a_0 \enskip x_0 \\
	a_1 \enskip x_1 \\
	... \\
	a_N \enskip x_N \\
\end{align*}

$a_i$~---~элемент массива, $x_i$~---~количество появления элемента в массиве. 
Если $x_i = 0$, то выводить $a_i$ не нужно. Порядок следования $a_i$~---~по возрастанию.

\SAMPLES

