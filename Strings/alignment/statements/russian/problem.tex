Рассмотрим язык, похожий на Си. В нём есть следующие простые типы:

\begin{itemize}
 \item \t{char}~--- занимает один байт
 \item \t{short}~--- занимает два байта
 \item \t{int}~--- занимает четыре байта
 \item \t{long}~--- занимает восемь байт
\end{itemize}

В языке есть оператор \t{sizeof}, который позволяет узнать размер любого типа в байтах.
К примеру, \t{sizeof(int)} равен четырём.

В языке есть оператор \t{alignof}, который позволяет узнать {\bf выравнивание} любого типа в байтах.
Адрес переменной какого-то типа \t{T} в памяти должен делиться на \t{alignof(T)}.
\t{sizeof(T)} всегда делится на \t{alignof(T)}.
\t{alignof(T)} всегда является неотрицательной целой степенью двойки.
Для простых типов, \t{alignof(T) == sizeof(T)}. 

В языке есть массивы фиксированной длины, состоящие из элементов одного типа.
Массив из \t{n} элементов, каждый типа \t{T} обозначается как \t{T[n]}.
\t{sizeof(T[n])} равен \t{sizeof(T) * n}. К примеру, \t{sizeof(short[13])} равен 26,
так как размер типа \t{short}~--- два байта, а в массиве 13 элементов.
\t{alignof(T[n])} равен \t{alignof(T)}.

В языке есть структуры~--- композитные типы, позволяющие объединять фиксированное количество переменных (полей) разных типов в одну.
Пусть в структуре $n > 0$ полей $f_1, \ldots, f_n$ типов $T_1, \ldots, T_n$.
Пусть эта структура лежит в памяти по адресу $a$.
Тогда должны выполняться следующие дополнительные условия:

\begin{itemize}
 \item Адрес $f_1$ равен $a$.
 \item Для $k = 2, \ldots, n$, Адрес $f_k$ больше адреса $f_{k-1}$.
 \item Поля не могут пересекаться
 \item
Как и для самой структуры, так и для всех её полей должны выполнятся стандартные правила выравнивания.
 \item Выравнивание структуры~--- максимум из выравниваний её полей.
 \item Размер структуры не меньше суммы размеров её полей.
 \item Размер структуры~--- минимальный из размеров, удовлетворяющий всем условиям.
\end{itemize}

Вам дано описание структуры \t{X}, состоящей из простых типов и одномерных массивов простых типов.
Найдите \t{sizeof(X)} и \t{alignof(X)}.

\InputFile

В первой строке входного файла записано натуральное число $n$~--- количество полей структуры \t{X}. $1 \le n \le 10^5$.

В последующих $n$ строках записаны типы полей. Каждый тип~--- это либо простой тип, либо одномерный массив из простых типов.

Гарантируется, что размер структуры \t{X} не превосходит одного гигабайта.

\OutputFile

На первой строке выведите одно целое число: \t{sizeof(X)}.

На второй строке выведите одно целое число: \t{alignof(X)}.

\SAMPLES
