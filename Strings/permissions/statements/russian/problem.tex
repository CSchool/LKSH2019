В файловую систему одного суперкомпьютера проник вирус, который сломал контроль
за правами доступа к файлам. Для каждого файла $N_i$ известно, с какими
действиями можно к нему обращаться: 
\begin{itemize}
\item запись (\texttt{W}),
\item чтение (\texttt{R}), 
\item запуск (\texttt{X}). 
\end{itemize}

Вам требуется восстановить контроль над правами доступа к файлам (ваша
программа для каждого запроса должна будет возвращать <<\texttt{OK}>> если над
файлом выполняется допустимая операция, или  же <<\texttt{Access denied}>>, если
операция недопустима).

\InputFile

В~первой строке входного файла содержится  число $N$ ($1 \le N \le
10\,000$)~--- количество файлов, содержащихся в данной файловой системе.

В следующих $N$ строчках содержатся имена файлов, состоящие из маленьких
латинских букв, цифр, точек и символов подчёркивания, и допустимых с ними
операций, разделенные пробелами. Длина имени файла не превышает $15$ символов.

Далее указано число $M$ ($1 \le M \le 50\,000$)~--- количество
запросов к файлам.

В~последних $M$ строках указан запрос вида <<Операция Файл>>. К одному и тому же
файлу может быть применено любое количество запросов.

\OutputFile

Для каждого из $M$ запросов нужно вывести в отдельной строке <<\texttt{Access denied}>> или <<\texttt{OK}>>.

\SAMPLES
