Шаблоном размера n назовем строку длины $n$, каждый из символов которой входит в множество \t{0}--\t{9}, \t{a}--\t{g}, \t{?}. Шаблоны преобразуются в строки из цифр по следующим правилам:
\begin{itemize}
\item символы от \t{0} до \t{9} могут быть преобразованы только сами в себя; 
\item символ \t{a} может преобразован в любой из символов \t{0}, \t{1}, \t{2}, \t{3} 
\item символ \t{b} может преобразован в любой из символов \t{1}, \t{2}, \t{3}, \t{4} 
\item символ \t{c} может преобразован в любой из символов \t{2}, \t{3}, \t{4}, \t{5} 
\item символ \t{d} может преобразован в любой из символов \t{3}, \t{4}, \t{5}, \t{6} 
\item символ \t{e} может преобразован в любой из символов \t{4}, \t{5}, \t{6}, \t{7} 
\item символ \t{f} может преобразован в любой из символов \t{5}, \t{6}, \t{7}, \t{8} 
\item символ \t{g} может преобразован в любой из символов \t{6}, \t{7}, \t{8}, \t{9}
\item символ \t{?} может преобразован в любой из символов от \t{0} до \t{9} 
\end{itemize}
Даны два шаблона: $p_1$ и $p_2$. Рассмотрим множество $S_1$ строк, которые могут быть получены из $p_1$ по описанным правилам, и множество $S_2$ строк, которые могут быть получены из $p_2$. Необходимо найти количество строк, входящих в оба этих множества. 

\InputFile

Первая строка входного файла содержит шаблон $p_1$, вторая --- шаблон $p_2$. Шаблоны имеют одинаковый положительный размер, не превосходящий $9$. 

\OutputFile

В выходной файл выведите ответ на задачу. 

\SAMPLES
