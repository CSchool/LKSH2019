Будем рассматривать только строчки, состоящие из заглавных латинских букв.
Например, рассмотрим строку \t{AAAABCCCCCDDDD}. Длина этой строки равна $14$.
Поскольку строка состоит только из латинских букв, повторяющиеся символы могут
быть удалены и заменены числами, определяющими количество повторений. Таким
образом, данная строка может быть представлена как \t{4AB5C4D}. Длина такой
строки $7$. Описанный метод мы назовем упаковкой строки. 

Напишите программу, которая берет упакованную строчку и восстанавливает по ней
исходную строку. 


\InputFile

Входной файл содержит одну упакованную строку. В строке могут встречаться
только конструкции вида \t{nA}, где $n$ --- количество повторений символа
(целое число от $2$ до $99$), а \t{A} --- заглавная латинская буква, либо
конструкции вида \t{A}, то есть символ без числа, определяющего количество
повторений. Максимальная длина строки не превышает $80$.

\OutputFile

В выходной файл выведите восстановленную строку. При этом строка должна быть
разбита на строчки длиной ровно по $40$ символов (за исключением последней,
которая может содержать меньше $40$ символов). 

\SAMPLES

\Note

В этой задаче запрещено использовать стандартные функции
языка для разбора данной строки на буквы и числа.
Считайте входные данные как одну строку и разбирайте её сами.
