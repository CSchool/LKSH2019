Таймер --- это часы, которые умеют подавать звуковой сигнал по прошествии некоторого периода времени. Напишите программу, которая определяет, когда должен быть подан звуковой сигнал.

\InputFile

В первой строке записано текущее время в формате \t{ЧЧ:ММ:СС} (с ведущими нулями). При этом оно удовлетворяет ограничениям: \t{ЧЧ} --- от $00$ до $23$, \t{ММ} и \t{СС} --- от $00$ до $60$.
Во второй строке записан интервал времени, который должен быть измерен. Интервал записывается в формате \t{Ч:М:С} (где \t{Ч}, \t{М} и \t{С} --- от $0$ до $10^9$, без ведущих нулей). Дополнительно если $\text{\t{Ч}}=0$ (или $\text{\t{Ч}}=0$ и $\text{\t{М}}=0$), то они могут быть опущены. Например, \t{100:60} на самом деле означает $100$~минут $60$~секунд, что то же самое, что \t{101:0} или \t{1:41:0}. А $42$ обозначает $42$~секунды. \t{100:100:100} --- $100$~часов, $100$~минут, $100$~секунд, что то же самое, что \t{101:41:40}.

\OutputFile

Выведите в формате \t{ЧЧ:ММ:СС} время, во сколько прозвучит звуковой сигнал. При этом если сигнал прозвучит не в текущие сутки, то дальше должна следовать запись \t{+<количество> days}. Например, если сигнал прозвучит на следующий день, то \t{+1 days}.

\SAMPLES
