Короткопалый бюльбюль Леонид из Якутии попал в довольно затруднительное
положение. В его уютном жилище поселился нежданный гость~--- детерминированный
конечный
автомат (ДКА). Как известно, ДКА допускает не каждое слово, и теперь Леониду нужно
тщательно следить за своей речью. В связи с этим у него возникла следующая задача:
определить, допускает ли данный ДКА заданное слово.

\InputFile

В первой строке входного файла находится слово, состоящее из не более чем $100\ 000$
строчных латинских букв.
Во второй строке содержатся целые числа $n$, $m$ и $k$~---количества состояний, переходов
и терминальных состояний в автомате соответственно. ($1 \le n, m \le 100\ 000$, $1 \le k \le n$).
В следующей строке содержатся $k$ целых чисел~--- номера терминальных состояний
(состояния пронумерованы от $1$ до $n$).
В следующих $m$ строках описываются переходы в формате <<a b c>>, где $a$~--- номер
исходного состояния перехода, $b$~--- номер состояния, в которое осуществляется переход,
и $c$~--- символ (строчная латинская буква), по которому осуществляется переход.
Стартовое состояние автомата всегда имеет номер $1$. Гарантируется, что из
любого состояния существует не более одного перехода по каждому символу.

\OutputFile

Требуется выдать строку <<\t{Accepts}>>, если автомат принимает заданное слово,
и <<\t{Rejects}>> в противном случае.

\SAMPLES
