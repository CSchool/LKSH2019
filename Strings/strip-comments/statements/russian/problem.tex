\epigraph{
\t{Q: Можно ли использовать в решении комментарии?} \\
\t{A: Без комментариев.}
}

Издревле почти в каждом монастыре ведутся летописи событий происходящих внутри
и за пределами самого монастыря. Не исключением является и Монастырь Светлой Луны.
Все свои наблюдения монахи тщательно записывали в особые дневники (Даарны).
Как часто случается, в этих летописях встречается не только описание реальных
событий, но и комментарии самого летописца. К счастью, в Монастыре Светлой Луны
был заведён порядок, что комментарии должны отделяться от описания событий одним
из следующих способов:

\begin{itemize}
    \item Комментарий начинается с <<\t{//}>> и продолжается до конца данной строки
          (символ перевода строки не является частью комментария).
    \item Комментарий начинается с <<\t{\{}>> и продолжается до ближайшего вхождения
          <<\t{\}}>>.
    \item Комментарий начинается с <<\t{/*}>> и продолжается до ближайшего вхождения
          <<\t{*/}>>.
\end{itemize}

Внутри комментария могут встречаться любые символы. Известно, что монахи
никогда не ошибаются и не оставляют комментарии незакрытыми. Также известно,
что после удаления комментария в тексте не возникнут новые комментарии.

По совету Наставника монахи хотят переписать все летописи, убрав из него
все комментарии. Ваша цель~--- помочь им в этом нелёгком деле.

\InputFile

Во входном файле содержится летопись длиной не более $10^6$ символов.
Каждая строка летописи не длиннее $250$ символов.

\OutputFile

Выведите летопись, очищенную от комментариев.

\SAMPLES
