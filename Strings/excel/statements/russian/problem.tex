Напишите программу, выполняющую функции очень простой электронной таблицы. Она работает с таблицей из $9$ строк от $1$ до $9$ и $26$ столбцов от $A$ до $Z$. Клетки таблицы обозначаются именами, составленными из кодов столбца и строки, например, $B1$, $S8$. 

Каждая клетка содержит выражение. Выражения используют целые константы, ссылки на клетки, скобки, бинарные операторы \t{+}, \t{--}, \t{*} и \t{/}~(целочисленное деление). Например, $567$, $E8/2$, $(3+B3)*(C4-1)$ являются правильными выражениями. Все операторы целочисленные. Деление на ноль даёт в результате ноль. 

Если значение ячейки, на которую ссылается некоторое выражение, не определено, оно считается равным нулю. Ситуация, когда две или более ячейки зависят друг от друга, является отдельным случаем --- циклической ссылкой. 

\InputFile

Первая строка содержит число выражений $N$. Следующие $N$ строк имеют формат \t{<Имя клетки>=<выражение>}. Все выражения корректные, и каждая ячейка определена не более чем одним выражением. 

Длина выражения в одной ячейке до $255$ символов, все аргументы и результаты меньше $1000000$. 

\OutputFile

В единственной строке выводится или значение клетки $A1$, или число $1000000$ (один миллион), если значение клетки $A1$ не может быть найдено из-за циклической ссылки. 

\SAMPLES
