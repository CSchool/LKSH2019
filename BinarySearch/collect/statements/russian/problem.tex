Как известно, Серёжа~--- ярый коллекционер бабочек.  Он имеет
огромную коллекцию, экспонаты которой собраны со всего мира.  Будем считать,
что в мире существует $2\,000\,000\,000$ видов бабочек. 

Чтобы не запутаться, Серёжа присвоил каждому виду уникальный номер.
Нумерация видов бабочек начинается с единицы.

Теперь он хочет знать, есть ли бабочка с видом $K$ в его коллекции, или же её
придётся добывать, затрачивая уйму сил и денег.

\InputFile

В первой строке входного файла содержится единственное число $N$ ($1 \leqslant
N \leqslant 100\,000$)~--- количество видов бабочек в коллекции Серёжи. 

В следующей строке через пробел находятся $N$ упорядоченных по возрастанию
чисел~--- номера видов бабочек в коллекции. 

Все виды бабочек в коллекции имеют различные номера.

В третьей строке файла записано число $M$ ($1 \leqslant M \leqslant
100\,000$)~--- количество видов бабочек, про которых Серёжа хочет узнать, есть
ли они у него в коллекции или же нет. В последней строке входного файла
содержатся через пробел $M$ чисел~--- номера видов бабочек, наличие которых
необходимо проверить.

\OutputFile

Выходной файл должен содержать $M$ строчек. Для каждого запроса выведите
``\texttt{YES}'', если бабочка с данным номером содержится в коллекции,
и ``\texttt{NO}''~--- в противном случае.

\SAMPLES
