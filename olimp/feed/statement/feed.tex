\begin{problem}{Колевская бухгалтерия}{feed.in}{feed.out}{2 секунды}{}

%Автор задачи: Демьянюк Виталий 
%Автор условия: Никита Кравцов

Бухгалтерия в виде Зазу работала при Муфасе и будет работать при Симбе. Но когда Шрам запер Зазу в клетку,
бухгалтерия ушла в вынужденный отпуск и потеряла все данные. И теперь Симба очень хочет их восстановить.

Бухгалтерия отвечала за количество антилоп и зебр в саванне. Кроме этого, она также считала количество останков животных. 
Одна единица останков образуется в результате поедания одной антилопы или зебры хищниками. Симба посчитал количество антилоп, 
зебр и останков в данный момент. Он точно знает, что с момента заточения Зазу не появилось ни одной новой зебры и ни одной новой 
антилопы, но некоторые зебры и антилопы могли быть съедены и превращены в останки. Кроме этого, Симба помнит, что в момент заточения 
Зазу антилоп было больше, чем зебр. Помогите ему посчитать количество различных наборов антилоп, зебр и останков, которые могли 
быть в момент заточения Зазу.

\InputFile
В строке заданы три целых числа $a$, $b$ и $c$ ($0 \le a, b, c \le 10^3$)~--- текущее количество антилоп, зебр и останков 
соответственно.

\OutputFile
В единственной строке выведите одно целое число $k$~--- количество возможных троек этих чисел в момент заточения Зазу.

\Examples
\begin{example}%
\exmp{
0 0 1
}{
1
}%
\exmp{
2 0 1
}{
3
}%
\end{example}

\Note
Ответом на первый тест из примера является единственная тройка чисел $(1, 0, 0)$.

Ответами на второй тест из примера являются следующие тройки чисел:
\begin{itemize}
\item$(2, 1, 0)$
\item$(3, 0, 0)$
\item$(2, 0, 1)$
\end{itemize}

\end{problem}
