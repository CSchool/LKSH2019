\begin{problem}{Ход конём}{стандартный ввод}{стандартный вывод}{1 секунда}{256 мегабайт}

На шахматной доске 8x8 указаны две различные клетки. Найдите кратчайший маршрут коня из первой клетки во вторую.

\InputFile
Во входном файле заданы координаты двух клеток. Каждая координата представлена двумя символами, где сначала указана строчная буква от \t{a} до \t{h}, а после буквы (без пробела) цифра от \t{1} до \t{8}, например \t{h8}. Каждая клетка записана в отдельной строке. Гарантируется, что координаты клеток различны.

\OutputFile
Программа должна вывести последовательность клеток, которые первая из которых совпадает с первой данной, а последняя совпадает со второй данной. Две соседние клетки должны быть соединены ходом коня, при этом количество клеток в последовательности должно быть минимально возможным. Если существует несколько ответов, разрешается вывести любой.

\Example

\begin{example}
\exmp{a1
b1
}{a1
b3
d2
b1
}%
\end{example}

\end{problem}

