\documentclass [11pt, a4paper, oneside] {article}

\usepackage [T2A] {fontenc}
\usepackage [utf8] {inputenc}
\usepackage [english, russian] {babel}
\usepackage {amsmath}
\usepackage {amssymb}
\usepackage #Language#{olymp}
\usepackage {comment}
\usepackage {epigraph}
\usepackage {expdlist}
\usepackage {graphicx}
\usepackage {ulem}
\usepackage {url}

\newcommand{\SAMPLES}{nothing here yet}

#Preamble#

\begin {document}

\contest
{#ContestName#}%
{#ContestLocation#}%
{#ContestDate#}%

\binoppenalty=10000
\relpenalty=10000

\renewcommand{\t}{\texttt}

\section*{Правила контеста}

В сегодняшней олимпиаде вам потребуется решать задачи, при этом минимизировать
размер исходного текста программы. За успешное решение задачи
вам будут начисляться баллы в количестве $\text{max}(0, 2\ 000 - S)$, где $S$~--- количество непробельных символов
в вашем исходном коде.
Штрафа за перепосылку нет, как и штрафа по времени.
Участники в табличке сортируются по сумме баллов.
Обратите внимание на то, что {\bf задачи не отсортированы по сложности}.
Удачи.

\newpage

#Statements#

\end {document}
