Изучая интернет-культуру, вы неожиданно для себя заметили удивительно частое
использование сочетания букв <<kek>>. Вы решили
измерить масштаб феномена, подсчитав количество букв в данном тексте, которые
являются частью одного из <<kek>>.

При подсчёте нужно учесть все подстроки <<kek>>, даже если они не являются
отдельными словами или пересекаются.

Например, в предложении:

\textit{why the kek u kek}

есть шесть букв, которые являются частью <<kek>>. Но в:

\textit{send kekeks to me plz}

только пять букв входят в подстроки <<kek>>.

\InputFile

В единственной строке записан отрывок из документа, длиной от 1 до 50 символов.
Строка может содержать только строчные латинские буквы '\t{a}'--'\t{z}' и пробелы ' '.

\OutputFile

Одно число~--- количество букв, которые являются частью одного из <<kek>>.

\SAMPLES
