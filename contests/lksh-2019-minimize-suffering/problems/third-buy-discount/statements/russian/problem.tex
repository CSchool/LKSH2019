\epigraph{\url{https://i.imgur.com/kK3AQyG.png}}

Сегодня в магазине ЛКШ <<Кэш>> проходит распродажа брелков.
Вы уже давно хотели приобрести себе парочку, а тут ещё и со
скидками! Каждый третий брелок вы покупаете с $d$-процентной скидкой!

Подсчитайте, какое минимальное количество электронных баллов вам понадобится,
чтобы купить все брелки.

\InputFile

В первой строке записано количество желаемых вами брелков $n$~($0 \le n \le 50$). 
В каждой из следующих $n$ строк записана цена $i$-го брелка в электронных баллах $p_i$~($0 \le p_i \le 2147483647$).

В последней строке записано целое число $d$~($0 \le d \le 100$)~--- скидка в процентах.

\OutputFile

Одно вещественное число с точностью не менее 5 знаков после запятой~---
минимальное количество денег, необходимое для того, чтобы купить все брелки.

\SAMPLES
