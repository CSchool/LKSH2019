Массив целых чисел называется \textit{стритом}, если в нем есть пять чисел,
которые являются пятью последовательными числами. Например, массив $\{ 6, 1, 9,
5, 7, 15, 8 \}$ является стритом, потому что он содержит $5$, $6$, $7$, $8$,
и $9$.

Для заданного массива найдите минимальное количество чисел, которое нужно
добавить в этот массив, чтобы тот стал стритом.

\InputFile

В первой строке находится число $n$~--- количество элементов массива ($1 \le
n \le 50$). В следующей строчке находятся $n$ целых чисел $a_i$ ($0 \le a_i \le
10^9$).

\OutputFile

Выведите единственное число~--- минимальное количество чисел, необходимое для
того, чтобы массив стал стритом.

\SAMPLES
