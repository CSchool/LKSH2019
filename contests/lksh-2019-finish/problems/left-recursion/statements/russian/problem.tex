В теории формальных грамматик и автоматов (ТФГиА) важную роль играют так называемые контекстно-свободные грамматики (КС-грамматики). КС-грамматикой будем называть четверку, состоящую из множества N нетерминальных символов, множества $T$ терминальных символов, множества $P$ правил (продукций) и начального символа $S \in N$. 

Каждая продукция $p \in P$ имеет форму $A \rightarrow R$, где $A$ нетерминальный символ ($A \in N$), а $R$ --- строка, состоящая из терминальных и нетерминальных символов. Процесс вывода слова начинается со строки, содержащей только начальный символ $S$. После этого на каждом шаге один из нетерминальных символов, входящих в текущую строку, заменяется на правую часть одной из продукций, в которой он является левой частью. Если после такой операции получается строка, содержащая только терминальные символы, что процесс вывода заканчивается. 

Во многих теоретических задачах удобно рассматривать так называемые нормальные формы грамматик. Процесс приведения грамматики к нормальной форме часто начинается с устранения левой рекурсии. В этой задаче мы будем рассматривать только ее частный случай, называемый непосредственной левой рекурсией. Говорят, что правило вывода $A \rightarrow R$ содержит непосредственную левую рекурсию, если первым символом строки $R$ является $A$. 

Задана КС-грамматика. Найти количество правил, содержащих непосредственную левую рекурсию. 


\InputFile

Первая строка входного файла содержит количество $n$ ($1 \le n \le 1000$) правил в грамматике. Каждая из последующих $n$ строк содержит по одному правилу. Нетерминальные символы обозначаются заглавными буквами латинского алфавита, терминальные --- строчными. Левая часть продукции отделяется от правой символами \t{->}. Правая часть продукции всегда непуста и имеет длину не более $30$ символов. 

\OutputFile

В выходной файл выведите ответ на задачу. 

\SAMPLES
