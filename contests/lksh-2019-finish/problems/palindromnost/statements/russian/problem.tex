Напомним, что палиндромом называется строка, которая читается одинаково как слева направо, так и справа налево. Например, палиндромами являются строки \t{abba} и \t{madam}. 

Для произвольной строки $s$ введем операцию деления пополам, обозначаемую $half(s)$. Значение $half(s)$ определяется следующими правилами: 

\begin{itemize}
\item Если $s$ не является палиндромом, то значение $half(s)$ не определено; 
\item Если $s$ имеет длину $1$, то значение $half(s)$ также не определено; 
\item Если $s$ является палиндромом четной длины $2m$, то $half(s)$ это строка, состоящая из первых $m$ символов строки $s$; 
\item Если $s$ является палиндромом нечетной длины $2m + 1$, большей $1$, то $half(s)$ это строка, состоящая из первых $m + 1$ символов строки $s$. 
\end{itemize}

Например, значения $half(\text{\t{informatics}})$ и $half(\text{\t{i}})$ не определены, $half(\text{\t{abba}}) = \text{\t{ab}}$, $half(\text{\t{madam}}) = \text{\t{mad}}$. 

Палиндромностью строки s будем называть максимальное число раз, которое можно применить к строке $s$ операцию деления пополам, чтобы результат был определен. 

Например, палиндромность строк \t{informatics} и \t{i} равна $0$, так как к ним нельзя применить операцию деления пополам даже один раз. Палиндромность строк \t{abba} и \t{madam} равна $1$, а палиндромность строки \t{totottotot} равна $3$, поскольку операция деления пополам применима к ней три раза: \t{totottotot} $\rightarrow$ \t{totot} $\rightarrow$ \t{tot} $\rightarrow$ \t{to}. 

Задана некоторая строка $s$ и число $k$. Необходимо вычислить палиндромность заданной строки. 

\InputFile

Первая строка содержит непустую строку $s$, состоящую из строчных букв латинского алфавита. Ее длина не превосходит $10^5$ символов. 

\OutputFile

Выведите в качестве ответа палиндромность строки, заданной во входных данных. 

\SAMPLES

